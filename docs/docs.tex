\documentclass[12pt]{extarticle}
\usepackage[a4paper,left=1cm,right=1cm,top=2cm,bottom=2cm]{geometry}

\usepackage{polyglossia}
\setmainlanguage{english}

\usepackage{fontspec}
\setmainfont{Lucida Grande}
\setsansfont{Lucida Sans}
\setmonofont{Lucida Console}

\usepackage{setspace}
\setstretch{1.25}

\usepackage{xcolor}
\usepackage{hyperref}
\hypersetup{%
    colorlinks=true,%
    urlcolor=blue,%
    urlbordercolor=blue,%
    linkcolor=blue,%
    linkbordercolor=blue%
}

\makeatletter
\Hy@AtBeginDocument{%
    \def\@pdfborder{0 0 1}
    \def\@pdfborderstyle{/S/U/W 1}
}
\makeatother

\usepackage{minted}
\usepackage{mdframed}
\surroundwithmdframed{minted}
\mdfsetup{backgroundcolor=gray!10}

\usepackage{amsmath}
\usepackage{multicol}

\newcommand{\code}[1]{\colorbox{gray!10}{\mintinline{bash}{#1}}}

\renewcommand{\arraystretch}{1.75}

\begin{document}
    \centerline{\bfseries\LARGE Acoustic}
    \bigskip
    Wide-anlge mode parabolic equation (WAMPE) solver for Helmholtz equation.
    \section{Getting started}
        \subsection{Requirements}
            The following is required to build Acoustic from scratch
            \begin{itemize}
                \item C++ compiler with C++17 support (\href{https://gcc.gnu.org/}{gcc} is recommended)
                \item \href{https://cmake.org/}{CMake}
                \item \href{https://www.alglib.net/}{ALGLIB}
                \item \href{https://www.boost.org/}{BOOST C++ Libraries}
                \item \href{https://github.com/nlohmann/json}{nlohmann::json}
                \item \href{http://www.fftw.org/}{fftw3}
                \item Included as git submodules
                \begin{itemize}
                    \item\href{https://github.com/Nauchnik/Acoustics-at-home/}{Acoustics-at-home}
                    \item\href{https://github.com/GoldFeniks/DORK}{DORK}
                    \item\href{https://github.com/GoldFeniks/delaunay}{delaunay}
                \end{itemize}
            \end{itemize}
        \subsection{Building}
        \par You can build ACOUSTIC with CMake as follows
        \begin{minted}{bash}
$ cd build/cmake
$ mkdir ACOUSTIC && cd ACOUSTIC
$ cmake -DCMAKE_BUILD_TYPE=Release ..
        \end{minted}
    \section{Usage}
        Solver utilizes command line interface as follows
        \begin{minted}{bash}
$ ACOUSTIC [task] [[option1] [option2] ...]
        \end{minted}
        \subsection{Tasks}
            \par Currently the following tasks are supported
            \begin{itemize}
                \item\code{solution (default)}\qquad Compute WAMPE solution
                \item\code{impulse}\qquad Compute acoustic impulse at receivers
                \item\code{modes}\qquad Compute wavenumbers and modal functions
                \item\code{rays}\qquad Compute acoustic rays
                \item\code{init}\qquad Compute initial conditions
            \end{itemize}
        \subsection{Options}
            \par Currently the following options are supported
            \subsubsection{General options}
                \begin{itemize}
                    \item\code{-h [ --help ]}\qquad Print help message
                    \item\code{-v [ --verbosity ] arg}\qquad Set verbosity level to \texttt{arg}. The following values are supported (bigger values include all previous)
                        \begin{itemize}
                            \item\code{0} nothing (default value)
                            \item\code{1} show execution time
                            \item\code{2} show progress bar
                            \item\code{3} print configuration information
                        \end{itemize}
                   \item\code{-r [ --report ] k}\qquad Only affects \code{solution} task. If verbosity level > 0 report every \code{k} computed rows. Prints nothing if set to \code{0} (default)
                   \item\code{-c [ --config ] arg}\qquad Specifies path to configuration file. Default is \code{config.json}
                \end{itemize}
            \subsubsection{Output options}
                \begin{itemize}
                    \item\code{-o [ --output ] filename}\qquad Specifies path to output file. Default is \code{output.txt}
                    \item\code{-s [ --step ] k}\qquad Output every \code{k}-th computed row. Default is \code{100}
                    \item\code{--binary} Switches to binary output
                \end{itemize}
            \subsubsection{Computational options}
                \begin{itemize}
                    \item\code{-w [ --workers ] n}\qquad Sets the number of threads for computation. Only affects \code{solution} and \code{impulse} tasks. Be default uses one thread
                    \item\code{-b [ --buff ] arg}\qquad Sets buffer size for multithreaded computations. Default is \code{100} 
                \end{itemize}
    \section{Configuration file}
        \par The configuration file is stored in JSON format with any of the following fields
        \subsection{General format}
            \par Real valued data is specified as one floating point value. Complex data is stored as two consecutive floating point values
        \subsection{Binary data types}
            \begin{itemize}
                \item\code{uint32}\qquad Unsigned 32-bit integer
                \item\code{double}\qquad IEEE 754 double value
                \item\code{complex}\qquad Two \code{double values}
            \end{itemize}
        \subsection{\label{sec:in-file}In-file data}
            \par Data stored in a file can be specified as follows
            \subsubsection{Text file}
                \begin{minted}{json}
{
    "field": [
        "text_file",
        "filename"
    ]
}
                \end{minted}
            \newpage
            \subsubsection{Binary file}
                \begin{minted}{json}
{
    "field": [
        "binary_file",
        "filename"
    ]
}
                \end{minted}
        \subsection{Table data\label{sec:table_data}}
            \subsubsection{Text data}
                \par The first row and column are coordinates values. The first coordinate follows columns, the second --- rows. Values should be (but are not required to) separated with spaces
                \begin{minted}{text}
0    y0 ...  yM
x0  v00 ... v0M
... ... ... ...
xN  vN0 ... vNM
                \end{minted}
            \subsubsection{Binary data}
                \par The binary table data is stored as follows
                \begin{table}[h]
                    \centering
                    \begin{tabular}{|c|l|l|}
                        \hline
                        \textbf{Type} & \multicolumn{1}{l|}{\textbf{Length}} & \textbf{Description}\\
                        \hline
                        \code{uint32} & \code{1} & Unsigned number \code{N}\\
                        \hline
                        \code{uint32} & \code{1} & Unsigned number \code{M}\\
                        \hline
                        \code{double} & \code{N} & First coordinate values\\
                        \hline
                        \code{double} & \code{M} & Second coordinate values\\
                        \hline
                        \code{double} & \code{NM} & Data values\\
                        \hline
                    \end{tabular}
                \end{table}
            \newpage
            \subsubsection{\label{sec:in-place}In-place data}
                \par Table data can be specified directly in a configuration file. \code{x}, \code{y} are first and second coordinate names respectively 
                \begin{minted}{json}
{
    "field": [
        "values",
        {
            "x": [],
            "y": [],
            "values": [ [], [], ]
        }
    ]
}
                \end{minted}
        \subsection{Floating-point fields}
            \begin{itemize}
                \item\code{"mode_subset"}\qquad Value in range \code{[-1, 1]}, used to truncate computed modes
                \item\code{"x0", "x1"}\qquad Domain border over \code{x} coordinate. \code{x0} is only used for ray starters otherwise is \code{0}
                \item\code{"y0", "y1"}\qquad Domain borders over \code{y} coordinate
                \item\code{"y_s", "z_s"}\qquad \code{y} and \code{z} coordinates of the source
                \item\code{"tolerance"}\qquad For impulse computation values less than \code{tolerance * max(spectre)} are skipped
                \item\code{"a0", "a1"}\qquad Min and max radian angles used for ray starters
                \item\code{"l0", "l1"}\qquad Min and max natural parameters used for ray starter
            \end{itemize}
        \subsection{Integer fields}
            \begin{itemize}
                \item\code{"max_mode"}\qquad Maximal number number of modes to use (-1 uses as many as there are)
                \item\code{"n_modes"}\qquad Use this many modes (\code{"max_mode"} still takes effect)
                \item\code{"nx", "ny"}\qquad Number of points over \code{x} and \code{y} coordinates. Only affects \code{solution} and \code{impulse} tasks
                \item\code{"ppm"}\qquad Number of points for modes computation
                \item\code{"mnx", "mny"}\qquad Number of points over \code{x} and \code{y} coordinates for modes computation
                \item\code{"ordRich"}\qquad Order of Richardson extrapolation
                \item\code{"n_layers"}\qquad Number of water layers
                \item\code{"past_n"}\qquad History length for transparent boundary conditions
                \item\code{"border_width"}\qquad Width of smoothed areas over left and rights domain borders. Should be less than \code{ny / 2}
                \item\code{"na"}\qquad Number of angular point for ray starters
                \item\code{"nl"}\qquad Number of natural parameter points for ray starter
            \end{itemize}
        \subsection{Boolean fields}
            \begin{itemize}
                \item\code{"complex_modes"}\qquad Uses complex-valued modes (accounts for attenuation)
                \item\code{"const_modes"}\qquad Modes are assumed to be \code{x}-independent
                \item\code{"additive_depth"}\qquad Add bottom layer depths instead of setting it
            \end{itemize}
        \subsection{Array fields}
            \par All following fields are real-valued
            \begin{itemize}
                \item\code{"betas"}\qquad Attenuation coefficients for \textbf{all} layers (water and bottom)
                \item\code{"bottom_layers"}\qquad Depths of bottom layers
                \item\code{"bottom_rhos"}\qquad Density of bottom layers
                \item\code{"bottom_c1s", "bottom_c2s"}\qquad Sound speed at the top and bottom of each bottom layer
                \item\code{"k0", "phi_s"}\qquad Wavenumbers and modal functions of the source. Both fields must be present to take effect
            \end{itemize}
        \subsection{Bathymetry}
            \par \code{"bathymentry"} specifies bottom depth of the domain and is given as \nameref{sec:table_data}. The coordinates names are \code{"x"} and \code{"y"}
        \subsection{Hydrolody}
            \par \code{"hydrology"} specifies sound speed in water over \code{"x"} and \code{"z"} coordinates as \nameref{sec:table_data}. Missing values can be specified as \code{-1}
        \subsection{Modes}
            \par \code{"modes"} is used to explicitly pass wavenumbers and modal functions to be used during computation.
            \subsubsection{In-file}
                \par Modal data can be specified as \nameref{sec:in-file} in either text or binary format
                \begin{itemize}
                    \item\code{N} --- the number of points over \code{x},
                    \item\code{M} --- the number of points over \code{y},
                    \item\code{K} --- the number of modes,
                    \item\code{k} --- wavenumber value,
                    \item\code{p} --- modal function value
                \end{itemize}
                \paragraph{\code{x}-independent}~\\
                    \par For \code{x}-independent modes the following format is used
                    \begin{minted}{bash}
M K
y0 ... yM
k00 ... k0M
... ... ...
kK0 ... kKM
p00 ... p0M
... ... ...
pK0 ... pKM
                    \end{minted}
                \paragraph{\code{x}-dependent}~\\
                    \par For \code{x}-dependent modes the following format is used
                    \begin{minted}{bash}
N M K
x0 ... xN
y0 ... yN
k000 ... k00M
k100 ... k10M
.... ... ....
kN0K ... kNMK
p000 ... p00M
.... ... ....
pN0K ... pNMK
                    \end{minted}
            \subsubsection{In-place}
                \par Modal data can also be specified as \nameref{sec:in-place}. For \code{x}-independent modes \code{"y"} can be omitted, \code{"k"} and \code{"phi"} are two-dimensional
                \begin{minted}{json}
{
    "modes": [
        "values",
        {
            "x": [],
            "y": [],
            "k": [ [ [], [], ], [], ],
            "phi": [ [ [], [], ], [], ]
        }
    ]
}
                \end{minted}
        \subsection{Receivers data}
            \par Receivers can be specified as \nameref{sec:in-file} or \nameref{sec:in-place} using \code{"receivers"} key as an array of tuples of three real values: \code{x}, \code{y} and \code{z} coordinates of the receiver. The first value of binary data must be \code{uint32} --- the number of receivers, text data must only contain coordinates.
        \subsection{Initial values}
            \par Initial data is specified using the \code{"init"} key. Currently \code{"green"}, \code{"gauss"}, \code{"ray_simple"} and \code{"ray"} values are supported
            \subsubsection{Green}
                \par The standard Greene starter
                \begin{equation*}
                    \mathcal{A}_j\left(0, y\right)=\frac{\varphi_j\left(z_s\right)}{2\sqrt{\pi}}\left(1.4467-0.8402k_{j,0}^2\left(y-y_s\right)^2\right)e^{-\frac{k_{j,0}^2\left(y-y_s\right)}{1.5256}}
                \end{equation*}
            \subsubsection{Gauss}
                \par The standard Gauss starter
                \begin{equation*}
                    \mathcal{A}_j\left(0, y\right)=\frac{\varphi_j\left(z_s\right)}{2\sqrt{\pi}}e^{-k_{j,0}^2\left(y-y_s\right)}
                \end{equation*}
            \subsubsection{Ray starters}
                \par Computes initial data using ray theory at \code{"x0"}. For \code{"ray_simple"} homogeneous medium is assumed and only source modes are used. For \code{"ray"} actual modes are used either dependent on \code{x} or not based on respective configuration parameter
        \subsection{Tapering}
            \par \code{"tapering"} is used to smooth ray starters edges.
            \begin{minted}{json}
{
    "tapering": {
        "type": {
            "value": 0,
            "left": 0,
            "right": 0
        }
    }
}
            \end{minted}
            The \code{"type"} can either be \code{"percentage"} or \code{"angled"}, and either \code{"value"} or both \code{"left"} and \code{"right"} must be present meaning percentage or angle range to be smoothed respectively
        \subsection{Coefficients}
            \par \code{"coefficients} for square root operator approximation can be specified as one of the following
            \begin{multicols}{2}
                \begin{minted}{json}
{
    "coefficients": [
        "pade"
    ]
}
                \end{minted}
                \columnbreak
                \begin{minted}{json}
{
    "coefficients": [
        "abs",
        {
            "a": 0,
            "b": 0,
            "c": 0
        }
    ]
}
                \end{minted}
            \end{multicols}
            \code{"pade"} being $a=1, b=0.75, c=0.25$
        \subsection{Default configuration}
            \par By default the following configuration is used and new values are either replace the old ones or added
            \begin{minted}{json}
{
    "mode_subset": -1,
    "max_mode": -1,
    "n_modes": 0,
    "ppm": 2,
    "ordRich": 3,
    "source_function": 25,
    "z_s": 100,
    "y_s": 0,
    "receivers": [
        "values",
        [
            [0, 0, 30]
        ]
    ],
    "n_layers": 1,
    "bottom_layers": [500],
    "bottom_c1s": [1700],
    "bottom_c2s": [1700],
    "bottom_rhos": [1.5],
    "betas": [0, 0.5],
    "complex_modes": true,
    "const_modes": true,
    "additive_depth": false,
    "past_n": 0,
    "border_width": 10,
    "bathymetry": [
        "values",
        {
            "x": [0, 1],
            "y": [0, 1],
            "values": [
                [200, 200],
                [200, 200]
            ]
        }
    ],
    "hydrology": [
        "values",
        {
            "x": [0, 1],
            "z": [0, 1],
            "values": [
                [1500, 1500],
                [1500, 1500]
            ]
        }
    ],
    "x0": 0,
    "x1": 15000,
    "nx": 15001,
    "y0": -4000,
    "y1": 4000,
    "ny": 8001,
    "coefficients": [
        "pade"
    ],
    "a0": -0.7854,
    "a1":  0.7854,
    "na": 90,
    "l0": 0,
    "l1": 4000,
    "nl": 4001,
    "init": "green",
    "tapering": {
        "angled": {
            "value": 0.1
        }
    },
    "tolerance": 0.02
}
            \end{minted}
\end{document}
